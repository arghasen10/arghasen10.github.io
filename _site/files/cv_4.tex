%%%%%%%%%%%%%%%%%%%%%%%%%%%%%%%%%%%%%%%%%
% Medium Length Professional CV
% LaTeX Template
% Version 2.0 (8/5/13)
%
% This template has been downloaded from:
% http://www.LaTeXTemplates.com
%
% Original author:
% Trey Hunner (http://www.treyhunner.com/)
%
% Important note:
% This template requires the resume.cls file to be in the same directory as the
% .tex file. The resume.cls file provides the resume style used for structuring the
% document.
%
%%%%%%%%%%%%%%%%%%%%%%%%%%%%%%%%%%%%%%%%%

%----------------------------------------------------------------------------------------
%	PACKAGES AND OTHER DOCUMENT CONFIGURATIONS
%----------------------------------------------------------------------------------------

\documentclass{resume} % Use the custom resume.cls style
\usepackage[colorlinks = true,
linkcolor = black,
urlcolor  = black,
citecolor = black,
anchorcolor = black]{hyperref}

\usepackage[left=0.75in,top=0.6in,right=0.75in,bottom=0.6in]{geometry} % Document margins
\newcommand{\tab}[1]{\hspace{.2667\textwidth}\rlap{#1}}
\newcommand{\itab}[1]{\hspace{0em}\rlap{#1}}
\name{Argha Sen} % Your name
\address{} % Your address
%\address{123 Pleasant Lane \\ City, State 12345} % Your secondary addess (optional)

\address{ \\ } 
% Your phone number and email

\begin{document}
	
	%----------------------------------------------------------------------------------------
	%	EDUCATION SECTION
	%----------------------------------------------------------------------------------------
	
	\textbf{Website:} \url{https://arghasen10.github.io/} \\
	\textbf{Github:} \url{https://github.com/arghasen10/} \\
	\textbf{Google Scholar:} \url{https://scholar.google.com/citations?user=QF2toEoAAAAJ&hl=en} \\
	\textbf{LinkedIn:} \url{https://www.linkedin.com/in/arghasen10/} \\
	
	\begin{rSection}{Research Experience}
		\begin{itemize}
			\item Experience with \textbf{mmWave and acoustic FMCW signal processing} techniques. I have used mmWave and accoustic FMCW sensors for localization, tracking and activity recognition of humans under indoor environments.
			\item Hands-on experience in \textbf{hardware prototyping} including circuit design, circuit debugging, deployment. Developed prototype hardwares for Embedded Pollution Sensors mounted on a drone for air quality assesment.
			\item Experience conducting \textbf{human research studies}. Conducted real-time driver inattenviness study using COTS mmWave Radars by collecting doppler shifts in the mmWave data because of drivers attentive body movements, such as talking, yawning, nodding, etc.
			\item Experience with design of \textbf{UAVs}, control systems, RF communication devices. Developed drones using ArduCopter APM flight controller board and mounted air quality measurement sensors.
			\item Experience with \textbf{Computer Networks}, IoT devices, Distributed Sensor Networks. Worked on energy optimisation in 5G Cellular Networks, using network simulator ns-3.
		\end{itemize}
		
		
	\end{rSection}
	
	\begin{rSection}{Education}
		
		{\bf Indian Institute of Technology Kharagpur} \hfill {\em Jan 2021 - present} 
		\\ Doctor of Philosophy\hfill { }
		\\ Department of Computer Science and Engineering
		
		
		{\bf National Institute of Technology Durgapur} \hfill {\em August 2016 - June 2020} 
		\\ Bachelor of Technology \hfill {CGPA: 8.97}
		\\ Department of Electronics and Communication Engineering  
		
		{\bf Higher Secondary School}  \hfill {\em 2016}
		\\Central Board Of Higher Secondary Education (CBHSE) \hfill { Aggregate Percentage: 95.4}
		\\Jawahar Navodaya Vidyalaya, Birbhum
		
		{\bf Secondary School}  \hfill {\em 2014}
		\\ Central Board Of Secondary Education (CBSE) \hfill { Cum. GPA: 9.8}
		\\Jawahar Navodaya Vidyalaya, Birbhum
		
		%Minor in Linguistics \smallskip \\
		%Member of Eta Kappa Nu \\
		%Member of Upsilon Pi Epsilon \\
		
		
	\end{rSection}
	
	\begin{rSection}{Patents}
		\begin{enumerate}
			\item \textbf{Argha Sen}, Anirban Das, Sandip Chakraborty ``\textbf{A system for in-vehicle passive monitoring of driver behaviors.}". \textbf{India Patent Application: TEMP/E-1/18612/2024-KOL, filed 2nd February 2024}.
		\end{enumerate} 
	\end{rSection}
	\newpage
	\begin{rSection}{Publications} 
		\begin{enumerate}
			\small
			\item \textbf{Argha Sen}, Bhupendra Pal, Seemanth Achari. 
			``\textbf{SmartMME: Implementation of Base Station Switching Off Strategy in ns-3r}". \textbf{Accepted in IEEE ANTS 2024}.
			
			\item Nuwan Bandara, Thivya Kandappu, \textbf{Argha Sen}, Ila Gokarn, Archan Misra.
			``\textbf{EyeGraph: Modularity-aware Spatio Temporal Graph Clustering for Continuous Event-based Eye Tracking}".
			\textbf{Accepted in NeurIPS Datasets and Benchmarks Track 2024}.
			
			\item \textbf{Argha Sen}, Nuwan Bandara, Ila Gokarn, Thivya Kandappu, Archan Misra.
			``\textbf{EyeTrAES: Fine-grained, Low-Latency Eye Tracking via Adaptive Event Slicing}".
			\textbf{Accepted in ACM IMWUT 2024}

			\item \textbf{Argha Sen}, Amrta Chaurasia, Avijit Mandal, Sandip Chakraborty. ``\textbf{Capturing Human Emotion Pervasively using COTS mmWave Radar}". \textbf{Accepted in ACM COMPASS Posters 2024}.
			
			\item Rajib Sarkar, \textbf{Argha Sen}, Sandip Chakraborty. ``\textbf{Live In-car Traffic Monitoring using mmWave Sensing}". \textbf{Accepted in ACM COMPASS Posters 2024}.
			
			\item \textbf{Argha Sen}, Avijit Mandal, Prasenjit Karmakar, Anirban Das, Sandip Chakraborty ``\textbf{Passive Monitoring of Dangerous Driving Behaviors Using mmWave Radar}". \textbf{Pervasive and Mobile Computing Journal (PMC)}.
			
			\item \textbf{Argha Sen}, Soham Chakraborty, Soham Tripathy, Sandip Chakraborty ``\textbf{Poster: Dynamic Ego-Velocity estimation Using Moving mmWave Radar: A Phase-Based Approach}". \textbf{Accepted in ACM MobiSys Posters 2024}
			\item \textbf{Argha Sen}, Anirban Das, Swadhin Pradhan, Sandip Chakraborty ``\textbf{Demo Abstract: \textit{MARS} -An mmWave-based Multi-user Activity Tracking Solution}". \textbf{Accepted in ACM/IEEE IPSN Demos 2024}
			
			\item \textbf{Argha Sen}, Anirban Das, Swadhin Pradhan, Sandip Chakraborty ``\textbf{Continuous Multi-user Activity Tracking via Room-Scale mmWave Sensing}". \textbf{Accepted in ACM/IEEE IPSN 2024}.
			
			\item \textbf{Argha Sen}, Avijit Mandal, Prasenjit Karmakar, Anirban Das, Sandip Chakraborty ``\textbf{mmdrive: mmWave Sensing for Live Monitoring and On-Device Inference of Dangerous Driving}". \textbf{IEEE Percom 2023}.
			
			\item \textbf{Argha Sen}, Anirban Das, Prasenjit Karmakar, Sandip Chakraborty ``\textbf{mmAssist: Passive Monitoring of Driver's Attentiveness Using mmWave Sensors}". \textbf{IEEE COMSNETS 2023}.
			
			\item \textbf{Argha Sen}, Ayan Zunaid, Soumyajit Chatterjee, Basabdatta Palit, Sandip Chakraborty ``\textbf{Revisiting Cellular Throughput Prediction: Learning \textit{in-situ} for Multi-device and Multi-network Considerations for 5G}". \textbf{EWSN 2023}
			
			\item Basabdatta~Palit, \textbf{Argha Sen}, Abhijit~Mondal, Ayan~Zunaid, Jay~Jayatheerthan, Sandip Chakraborty ``\textbf{Improving UE Energy Efficiency through Network-aware Video Streaming over 5G}". \textbf{IEEE Transactions on Network and Service Management}
			\item Praveen Kumar Sharma, Bidyut Dalal, Ananya Mondal, \textbf{Argha Sen}, Amartya Banerjee, Sandip Mondal, Tanmay De, Sujoy Saha ``\textbf{Indoor Air Sensing: A Study in Cost, Energy, Reliability and Fidelity in Sensing}". \textbf{Sensing and Imaging Journal, Springer 2023}
			\item \textbf{Argha Sen}, Sashank Bonda, Jay Jayatheerthan, Sandip Chakraborty ``\textbf{Implementation of mmWave-energy Module and Power Saving Schemes in ns-3}". \textbf{WNS3 2022}.
			\item \textbf{Argha Sen}, Sashank Bonda, Jay Jayatheerthan, Sandip Chakraborty``\textbf{An ns3-based Energy Module for 5G mmWave Base Stations}" \textbf{IEEE INFOCOM Posters 2022}
			\item \textbf{Argha Sen}, Abhijit Mondal, Basabdatta Palit, Jay Jayatheerthan, Krishna Paul, Sandip Chakraborty ``\textbf{An ns3-based Energy Module of 5G NR User Equipments for Millimeter Wave Networks}" \textbf{IEEE INFOCOM Posters 2021}
			\item Praveen Kumar Sharma, Suraj Gupta, \textbf{Argha Sen}, Tanmoy De, Sujoy Saha ``\textbf{Exploring Collision Avoidance during Communication Over Sound for Healthy Environment}" \textbf{SoCIeTY, ICDCN Workshops 2020}
			\item \textbf{Argha Sen}, Monsij Biswal, Shreyan Datta ``\textbf{Intelligent Traffic Routing Based on Real-time Congestion Analysis}" \textbf{IEEE INDICON 2019}
			
		\end{enumerate}
	\end{rSection}
	
	\begin{rSection}{Research Projects}
		\begin{rSubsection}{}{}{}{}
			\item \textbf{Real-time Multi-User Localization, Tracking, and Activity Recognition using mmWave}\\
			\textbf{Abstract:} With the recent advancement in 5G technology which builds on top of mmWave-based communication, we observe a massive paradigm shift towards a new form of RF sensing where millimeter-level range-resolution can be achieved in localizing a human subject or in determining their activities. In this project, we study the extent to which we can achieve high granularity mmWave sensing under diverse scenarios such as single or multi-user, macro, or micro-user activities. Finally, we implement a system to accurately localize and track multiple users inside a room and recognize their activities. The system is robust and monitors human activity opportunistically in real time under environment-independent scenarios. \\
			\textbf{GitHub:} \url{https://github.com/arghasen10/mars}
			
			\item \textbf{mmWave Sensing for Live Monitoring and On-Device Inference of Dangerous Driving}\\
			\textbf{Abstract:} In this work, we explore the feasibility of purely using mmWave radars to detect dangerous driving behaviors. We first study characteristics of dangerous driving and find some unique patterns of range-doppler caused by 9 typical dangerous driving actions. We then develop a Fused-CNN model to detect dangerous driving instances from regular driving and classify 9 different dangerous driving actions. Through extensive experiments with 5 volunteer drivers in real driving environments, we observe that our system can distinguish dangerous driving actions with an average accuracy of 97
			2\%. We also compare our approach with existing state-of-the-art baselines to establish its significance. \\
			\textbf{GitHub:} \url{https://github.com/arghasen10/mmdrive}
			
			
		\end{rSubsection}
	\end{rSection}
	
	
	%----------------------------------------------------------------------------------------
	%	WORK EXPERIENCE SECTION
	%----------------------------------------------------------------------------------------
	\begin{rSection}{Work Experience}
		
%		\begin{rSubsection}{Research Scholar}  
%			
%			
%			 
%			
%		\end{rSubsection}
		\begin{rSubsection}{Visiting Postgraduate Research Student, SMU, Singapore}  
			{Aug 2023 - Feb 2024}{}{}
%			\item High-Frequency Temporal Eye Movement Features For Robust Biometric Authentication Using Event Cameras
			\item HyGeNe: Efficient Video Compression Using Hybrid Artificial-Spiking Neural Networks\\
		\end{rSubsection}
		\begin{rSubsection}{Teaching Assistant}{Jan 2024 - Apr 2024}{}{}
			\item NPTEL, Sub: Computer Networks \& Internet Protocol
		\end{rSubsection}
		
		\begin{rSubsection}{Teaching Assistant}{Jan 2023 - Apr 2023}{}{}
			\item NPTEL, Sub: Computer Networks \& Internet Protocol
		\end{rSubsection}
		
		\begin{rSubsection}{Junior Research Fellow (INTEL sponsored Project)}  
			{Oct 2020 - April 2023}{}{}
				\item Traffic Engineering for Enabling Energy-aware Design in Next Generation Cellular Networks
				\item mmwave-energy module for ns-3 (Completed)\\
				\textbf{GitHub:} \url{https://github.com/arghasen10/mmwave-energy}
		\end{rSubsection}
		
		\begin{rSubsection}{Teaching Assistant}{Jan 2022 - Apr 2022}{}{}
				\item NPTEL, Sub: Computer Networks \& Internet Protocol
		\end{rSubsection}
		
		
		\begin{rSubsection}{Summer Research Intern}{May 2019 - July 2019}{Integrated Test Range, Chandipur, DRDO, Govt. of India}{}{}
				\item 3D Tracking and Geo-Localization of a target using Unmanned Aerial Vehicles\\
		\end{rSubsection}
		
		\begin{rSubsection}{Winter Research Intern}{Nov 2018 - Jan 2019}{Mobile Computing \& Network Research Group MCNRG, NIT Durgapur}{}{}
				\item AeT-Drone: Aerial Environment Sensing and Traffic Surveillance using sensor-enabled Drone/UAV\\
								\textbf{GitHub:} \url{https://github.com/arghasen10/Image-Processing-in-UAV} \\
				\textbf{Video:} \url{https://www.youtube.com/watch?v=t7wGiFjItSI}
		\end{rSubsection}
		
		\begin{rSubsection}{Summer Intern}{May 2018 - July 2018}{Criotam Technologies PVT LTD, Bangalore}{}{}
		
				\item Designed and developed three IoT prototypes for Sports Authority of India. (Starting Block, Force Plate, Timing Gates)
				\item Developed a Facial Recognition system as a LockOut-Tagout (LOTO)
				system for Industrial IoT. \\
				\textbf{Video:} \url{https://www.youtube.com/channel/UCEuhgjsfn7CaBQ2TpSGfjUw}
		
		\end{rSubsection}
		
	\end{rSection}
	
	%\vspace{50mm}
	%	EXAMPLE SECTION
	%----------------------------------------------------------------------------------------
	
	
	
	\begin{rSection}{Achievements} 
		\begin{itemize}
			\item Recipient of ACM COMPASS Travel Grant for attending ACM COMPASS 2024
			\item Recipient of ACM MobiSys Student Travel Grant for attending ACM MobiSys 2024
			\item Recipient of ACM IARCS Travel Grant for attending IPSN 2024
			\item Recipient of COMSNETS LRN Travel Grant for attending IEEE PerCom 2023
			\item Recipient of ACM IARCS Travel Grant for attending IEEE PerCom 2023
			\item Recipient of IEEE INFOCOM Travel Grant for attending IEEE INFOCOM 2021
			\item Best Project Award at ITR Chandipur, DRDO Project title: 3D Tracking and geolocalization of the target using Unmanned Aerial Vehicles
			\item Participated in Festival of Innovation and Entrepreneurship (FINE) 2019 at NIF Gandhinagar, Gujarat from March 15 to 18, 2019.
			\item 1st Position in Techmela, The annual science exhibition organized in AAROHAN 2019. AeT-Drone: Aerial Environment Sensing and Traffic Monitoring with Drone(UAV) complete working prototype presented in Techmela.
			\item 1st Position in Onspot IoT Hackathon organized in AAROHAN 2019: The first position among 70 entries in IoT Hackathon. An intrusion detection system with a Raspberry Pi Zero working prototype is presented.
			\item IoT Hackathon Winner: 1st Position in Hackoverflow, IoT Hackathon organized in Aavishkar 2018, Techno-Management fest of NIT Durgapur by GNU LINUX USER's GROUP and HackerEarth.
			\item District and School Topper in Higher Secondary Examination, CBHSE.
			\item Secured 15883 Rank in JEE Main 2016, out of 12,07,058
			applicants all over India.
		\end{itemize}
	\end{rSection}
	\newpage
	\begin{rSection}{Technical Skills}
		\textbf{Strongest Areas} - Sensors, Signal Processing, Machine Learning, Embedded Systems, Internet Of Things, Computer networks, Unmanned Aerial Vehicles, Raspberry Pi, Arduino, UDOO Neo
		
		\textbf{Languages} - Python, C, C++, UNIX Shell, JavaScript, Java, HTML, Android.
		
		\textbf{Tools and Frameworks} - MmWave Studio, Matlab, Git, Android Studio, ns3, Mininet
		
		\textbf{Operating Systems} - Linux: Ubuntu, Cent OS, Windows, Android\\
	\end{rSection}
	
	\begin{rSection}{POSITION OF RESPONSIBILITY}
		
		{\bf Vice Captain of Team Robocon, NIT Durgapur} - Sep 2018-Mar 2019 \hfill {\em NIT Durgapur} 
		
		{\bf Teaching Assistant; Computer Networks} - Jan 2023-April 2023\hfill {\em IIT Kharagpur}
		
		{\bf Teaching Assistant; Advances in Operating Systems} - July 2022-Nov 2022\hfill {\em IIT Kharagpur}
		
		{\bf Teaching Assistant; PDS} - Jan 2021-July 2022\hfill {\em IIT Kharagpur} 
		
		
		%------------------------------------------------
		
	\end{rSection}
	
%	\newpage
	\begin{rSection}{Personal Details}
		\textbf{Date of Birth:} 20th November, 1998 \\
		\textbf{Gender:} Male \\ 
		\textbf{Nationality:} Indian\\
		\textbf{Permanent Address:} Vill+P.O. Chatra, Rampurhat, Birbhum, West Bengal. PIN: 731238\\
		\textbf{Phone:} (+91) 7001927155
		
	\end{rSection}
	
	\begin{rSection}{References}
		
		\item[\#] {\bf Dr. Sandip Chakraborty} \hfill {\em Associate Professor} \\
		Department of Computer Science \& Engineering \smallskip \\
		Indian Institute of Technology Kharagpur , West Bengal, India.\\
		Institute Website: http://www.iitkgp.ac.in \\
		E-mail: sandipc@cse.iitkgp.ac.in / sandipchkraborty@gmail.com \\Ph: +91 (3222) 282898 
		
		\item[\#] {\bf Prof. Archan Misra} \hfill {\em Professor, Vice Provost} \\
		School of Computing and Information Systems \smallskip \\
		Simgapore Management University, Singapore\\
		E-mail: archanm@smu.edu.sg \\Ph: +65 6808 5202     
		
		\item[\#] {\bf Dr. Soumyajit Chatterjee } \hfill {\em Research Scientist} \\
		Nokia Bell Labs\smallskip \\
		Cambridge , United Kingdom.\\
		E-mail: sjituit@gmail.com
		
		\item[\#] {\bf Dr. Thivya Kandappu} \hfill {\em Assistant Professor} \\
		School of Computing and Information Systems \smallskip \\
		Simgapore Management University, Singapore\\
		E-mail: thivyak@smu.edu.sg \\Ph: +65 68085446     
		
		\item[\#] {\bf Manvendra Singh Chauhan  } \hfill {\em Scientist C} \\
		Electro Optical Tracking System \smallskip \\
		Integrated Test Range (DRDO) , Chandipur, Odisha , India.\\
		E-mail: manvendresc@gmail.com
		
				\item[\#] {\bf Prof. Sujoy Saha} \hfill {\em Assistant Professor} \\
		Department of Computer Science \& Engineering \smallskip \\
		National Institute of Technology Durgapur , West Bengal, India.\\
		Institute Website: http://nitdgp.ac.in/\\
		E-mail: sujoy.ju@gmail.com 
	\end{rSection}
	
	
\end{document}
